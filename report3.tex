\documentclass[12pt,a4paper]{jsarticle}

\usepackage{amssymb,amsmath,amsfonts,color}
\usepackage[dvipdfmx]{graphicx}
\usepackage{float}
%\usepackage{listings}
%\usepackage[bold]{otf}
\usepackage{color}
\usepackage{fancyhdr}
\usepackage{lastpage}
\usepackage{caption}

\if0
\pagestyle{fancy}
{%
\fancyhf{} 
\cfoot{\thepage/\protect\pageref{LastPage}}
\lhead{\scriptsize 情報数学II - 第 3 章レポート -}
\rhead{\scriptsize 学籍番号:1111  氏名:mesnwako}
\renewcommand{\headrulewidth}{16pt}
}
\pagestyle{fancy}
\fi

\title{\Huge 情報数学II - 第 3 章レポート-}
\author{\textbf{1111  meswanko} \\\\ (協力者: )}
\date{\fontsize{7pt}{0pt}-2017 年 7 月 10 日-\selectfont}

\begin{document}
\maketitle

\thispagestyle{empty}
\if0
\section*{はじめに}
\begin{itemize}
 \item[☆] \mbox{} \\
\end{itemize}
\fi
\pagebreak



\setcounter{page}{1}
\section*{問題3 - 1}
\begin{enumerate}
 \item[1]  \mbox{} \\
 %%追加説明あり
 %%わからんからmath-nakagawa〜〜おせーて♡
 \item[2]  \mbox{} \\
 %%解いた
 \begin{align*}
 (42899634253)_{12} &= 4 \times 12^{10} + 2 \times 12^{9} + 8 \times 12^{8} \\
  & + 9 \times 12^{7} + 9 \times 12^{6} + 6 \times 12^{5} +3 \times 12^{4} \\
  & + 4 \times 12^{3} + 2 \times 12^{2} + 5 \times 12^{1} + 3 \\
 \end{align*}
 また, $12^{k} \equiv 1(mod 11)$ より
 \begin{align*}
 (42899634253)_{12}
 &\equiv 4 + 2 + 8 + 9 + 9 + 6 \\
 & + 3 + 4 + 2 + 5 +3 (mod~11) \\
 &\equiv 55 \\
 &\equiv 0 (mod~11) \\ 
 \end{align*}
 より, $(42899634253)_{12}$ は $(11)_{10}$ で割り切れる.
 \item[3]  \mbox{} \\
 %%解いた
 \begin{align*}
 (3A6F2B1)_{16} &= (3 \times 16^{6} + 10 \times 16^{5} + 6 \times 16^{4} \\
  & + 15 \times 16^{3} + 2 \times 16^{2} + 11 \times 16^{1} + 1)_{10}\\
 \end{align*}
 また, $(16)_{10}^{k} \equiv 1(mod 15)$ より
 \begin{align*}
 (3A6F2B1)_{16}
 &\equiv (3 + 10 +6 + 15 + 2 + 11 + 1)_{10} \\
 &\equiv (48)_{10} \\
 &\equiv (3)_{10} (mod~15) \\ 
 &\equiv (3)_{16} (mod~F) \\
 \end{align*}
 \item[4]  \mbox{} \\
 \item[5]  \mbox{} \\
 \item[6]  \mbox{} \\
 \item[7]  \mbox{} \\
 \item[8]  \mbox{} \\
 %%追加説明あり
 \item[9]  \mbox{} \\
 \item[10]  \mbox{} \\
\end{enumerate}
\pagebreak

\section*{問題3 - 2}
\begin{enumerate}
 \item[1]  \mbox{} \\
 %%これのみはだめ
 \item[2]  \mbox{} \\
 %%解いた
 $\left\{ 59, 162, -353, 107, 77, -50, 116 \right\}$ より,
 \begin{align*}
 59 &\equiv 3 (mod~7) \\
 162 &\equiv 2 (mod~7) \\
 -353 &\equiv 4 (mod~7) \\
 107 &\equiv 2 (mod~7) \\
 77 &\equiv 0 (mod~7) \\
 -50 &\equiv 6 (mod~7) \\
 116 &\equiv 4 (mod~7) 
 \end{align*}
 より, $7$ を法とする\underline{完全代表系とはならない}.
 \item[3]  \mbox{} \\
 %%解いた
 $\left\{ -141, 65, 103, 70, -6, 199, 32 \right\}$ より,
 \begin{align*}
 -141 &\equiv 6 (mod~7) \\
 65 &\equiv 2 (mod~7) \\
 103 &\equiv 5 (mod~7) \\
 70 &\equiv 0 (mod~7) \\
 -6 &\equiv 1 (mod~7) \\
 199 &\equiv 3 (mod~7) \\
 32 &\equiv 4 (mod~7) 
 \end{align*}
 より, $7$ を法とする\underline{完全代表系とはなる}.
 \item[4]  \mbox{} \\
 \item[5]  \mbox{} \\
\end{enumerate}
\pagebreak

\section*{問題3 - 3}
\begin{enumerate}
 \item[1]  \mbox{} \\
 %%わからんからmath-nakagawaおせーて
 \item[2]  \mbox{} \\
 \item[3]  \mbox{} \\
 \item[4]  \mbox{} \\
 %%reduce
 \item[5]  \mbox{} \\
 \item[6]  \mbox{} \\
 %%reduce
 \item[7]  \mbox{} \\
 %%わからんーーmath-nakagawaおせーて。あのね、さいごどうすればいいかわからん
 \begin{align*}
  \begin{cases}
  x \equiv c_1 (mod~2) & \\
  x \equiv c_2 (mod~3) & \\
  x \equiv c_3 (mod~5) & \\
  x \equiv c_4 (mod~7)
  \end{cases}
 \end{align*}
 より, 第 $1$ 式を満たす整数は,
 \[ x = c_1 + 2r~ (A) \]
 の形である. これが第 $2$ 式を満たすように $r$ を定める. \\
 $(A)$ を第 $2$ 式に代入すると,
 \[ c_1 + 2r \equiv c_2 (mod~3) \] 
 であるから, そのためには $d_1 \equiv c_1 - c_2 (mod~3)$ として $r$ を
 \[ 2r \equiv d_1~ (B) \]
 を満たすようにとればよい. $(2, 3) = 1$ より, $(B)$ の解は $3$ を法としてただ $1$ つ存在する. それを $r = r_1 (mod~3)$ とすると, $(B)$ を満たす整数は,
 \[ r = r_1 + 3s \]
 の形である. したがってこれを $(A)$ に代入し, $x_2 = c_1 + 2r_1$ とおくと,
 \[ x = x_2 + 2 \times 3 \times s = x_2 + 6s \]
 が 第 $1$ 式, 第 $2$ 式をともに満たす整数となる. \\
 次にこれが第 $3$ 式を満たすように $s$ を定める. そして上と同様にして 第 $1$ 式から第 $3$ 式までを満たす整数が, 
 \[ x = x_3 + 2 \times 3 \times 5 \times t = x_3 + 30t \] 
 の形でえられる. ($x_3 = x_2 + 6s_1$)\\
 さらにこれが第 $4$ 式を満たすように $t$ を定める. そして上と同様にして 第 $1$ 式から第 $4$ 式までを満たす整数が, 
 \[ x = x_4 + 2 \times 3 \times 5 \times 7 \times u = x_4 + 210u \] 
 の形でえられる. ($x_4 = x_3 + 30t_1$) \\
 したがって, 求める解は,
 \begin{align*}
 x
 &\equiv x_4 (mod~210)\\
 &\equiv x_3 + 30t_1 (mod~210) \\
 &\equiv x_2 + 6s_1 + 30t_1 (mod~210) \\
 &\equiv c_1 + 2r_1 + 6s_1 + 30t_1 (mod~210) \\
 \end{align*}
 \item[8]  \mbox{} \\
 \item[9]  \mbox{} \\
 \item[10]  \mbox{} \\
 \item[11]  \mbox{} \\
\end{enumerate}
\pagebreak

\section*{問題3 - 4}
\begin{enumerate}
 \item[1]  \mbox{} \\
 %%解いた
 \begin{itemize}
 \item[(i)]
 \begin{align*}
 3^{47} &\equiv (3^3)^{15} \times 3^2 \\
      &\equiv 4^{15} \times 3^2 ~(\because 3^3 \equiv 27 \equiv 4 (mod~23) \mbox{より}) \\
      &\equiv (4^3)^5 \times 3^2 \\
      &\equiv 18^5 \times 3^2 ~(\because 4^3 \equiv 64 \equiv 18 (mod~23) \mbox{より}) \\
      &\equiv (-5)^5 \times 3^2 ~(\because 18 \equiv -5 (mod~23) \mbox{より}) \\
      &\equiv \left\{(-5)^2 \right\}^2 \times (-5) \times 3^2 \\
      &\equiv 2^2 \times (-5) \times 3^2 ~(\because (-5)^2 \equiv 25 \equiv 2 (mod~23) \mbox{より}) \\
      &\equiv -19 \\
      &\equiv 4 (mod~23)
 \end{align*}
 \item[(ii)]
 \begin{align*}
 7^{1000} &\equiv (7^2)^{500} \\
      &\equiv 1^{500} ~(\because 7^2 \equiv 49 \equiv 1 (mod~24)\mbox{より}) \\
      &\equiv 1 (mod~24)
 \end{align*}
 \end{itemize}
 \item[2]  \mbox{} \\
 \item[3]  \mbox{} \\
 \item[4]  \mbox{} \\
 \item[5]  \mbox{} \\
 \item[6]  \mbox{} \\
 \item[7]  \mbox{} \\
 \item[8]  \mbox{} \\
 %%結果をReduceに用いる
 \item[9]  \mbox{} \\
 \item[10]  \mbox{} \\
 \item[RSA]  \mbox{} \\
\end{enumerate}
\pagebreak

\section*{問題3 - 5}
\begin{enumerate}
 %\item[1]  \mbox{} \\
 %%除外
 \item[2]  \mbox{} \\
 \item[3]  \mbox{} \\
 \item[4]  \mbox{} \\
 \item[5]  \mbox{} \\
 \item[6]  \mbox{} \\
\end{enumerate}
\pagebreak

\section*{問題3 - 6}
\begin{enumerate}
 \item[1]  \mbox{} \\
 \item[2]  \mbox{} \\
 \item[3]  \mbox{} \\
 \item[4]  \mbox{} \\
 \item[5]  \mbox{} \\
 \item[6]  \mbox{} \\
 \item[7]  \mbox{} \\
 \item[8]  \mbox{} \\
 \item[9]  \mbox{} \\
 \item[10]  \mbox{} \\
 \item[11]  \mbox{} \\
\end{enumerate}
\pagebreak

\section*{問題3 - 6}
\begin{enumerate}
 \item[FGLM]  \mbox{} \\
 \item[定理証明]  \mbox{} \\
\end{enumerate}

\end{document}
