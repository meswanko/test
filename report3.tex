\documentclass[12pt,a4paper]{jsarticle}

\usepackage{amssymb,amsmath,amsfonts,color}
\usepackage[dvipdfmx]{graphicx}
\usepackage{float}
%\usepackage{listings}
%\usepackage[bold]{otf}
\usepackage{color}
\usepackage{fancyhdr}
\usepackage{lastpage}
%\usepackage{caption}

\if0
\pagestyle{fancy}
{%
\fancyhf{} 
\cfoot{\thepage/\protect\pageref{LastPage}}
\lhead{\scriptsize 情報数学II - 第 3 章レポート -}
\rhead{\scriptsize 学籍番号:1111  氏名:mesnwako}
\renewcommand{\headrulewidth}{16pt}
}
\pagestyle{fancy}
\fi

\title{\Huge 情報数学II - 第 3 章レポート-}
\author{\textbf{1111  meswanko} \\\\ (協力者: )}
\date{\fontsize{7pt}{0pt}-2017 年 7 月 10 日-\selectfont}

\begin{document}
\maketitle

\thispagestyle{empty}
\if0
\section*{はじめに}
\begin{itemize}
 \item[☆] \mbox{} \\
\end{itemize}
\fi
\pagebreak



\setcounter{page}{1}
\section*{問題3 - 1}
\begin{enumerate}
 \item[1]  \mbox{} \\
 %%追加説明あり
	$a = a_0 + a_1 \cdot 10 + \cdots + a_n \cdot 10^n$ を整数 $a > 0$ の10進数表示とする. \\
	$10 \equiv 1 \pmod{3}$ より, $10^k \equiv 1 \pmod{3}$ が得られるから,
	\[ a \equiv a_0 + a_1 + \cdots + a_n \pmod{3} \]
	したがって, $a \equiv 0 \pmod{3} \Leftrightarrow a_0 + a_1 + \cdots + a_n \equiv 0 \pmod{3}$.
 \item[1] 追加問題 \\
	\begin{itemize}
	 \item 7 の倍数 \\
		$10^3 \equiv -1 \pmod{7}$ より, $10^{3k} \equiv {(-1)}^k \pmod{7}$ が得られるから,
		\[ a \equiv a_0 a_1 a_2 - a_3 a_4 a_5 + a_6 a_7 a_8 - \cdots \pmod{7} \]
		したがって, $a \equiv 0 \pmod{7} \Leftrightarrow a \equiv a_0 a_1 a_2 - a_3 a_4 a_5 + a_6 a_7 a_8 - \cdots \pmod{7}$.
	 \item 13 の倍数 \\
		$10^3 \equiv -1 \pmod{13}$ より, $10^{3k} \equiv {(-1)}^k\pmod{11}$ が得られるから,
		\[ a \equiv a_0 a_1 a_2 - a_3 a_4 a_5 + a_6 a_7 a_8 - \cdots \pmod{11} \]
		したがって, $a \equiv 0 \pmod{7} \Leftrightarrow a \equiv a_0 a_1 a_2 - a_3 a_4 a_5 + a_6 a_7 a_8 - \cdots \pmod{11}$.
	 \item 37 の倍数 \\
		$10^3 \equiv 1 \pmod{37}$ より, $10^{3k} \equiv 1 \pmod{37}$ が得られるから,
		\[ a \equiv a_0 a_1 a_2 + a_3 a_4 a_5 + a_6 a_7 a_8 + \cdots \pmod{37} \]
		したがって, $a \equiv 0 \pmod{37} \Leftrightarrow a \equiv a_0 a_1 a_2 + a_3 a_4 a_5 + a_6 a_7 a_8 + \cdots \pmod{37}$.
	\end{itemize}
 \item[2]  \mbox{} \\
 %%解いた
 \begin{align*}
 (42899634253)_{12} &= 4 \times 12^{10} + 2 \times 12^{9} + 8 \times 12^{8} \\
  & + 9 \times 12^{7} + 9 \times 12^{6} + 6 \times 12^{5} +3 \times 12^{4} \\
  & + 4 \times 12^{3} + 2 \times 12^{2} + 5 \times 12^{1} + 3 \\
 \end{align*}
 また, $12^{k} \equiv 1 \pmod{11}$ より
 \begin{align*}
 (42899634253)_{12}
 &\equiv 4 + 2 + 8 + 9 + 9 + 6 \\
 & + 3 + 4 + 2 + 5 +3 \pmod{11} \\
 &\equiv 55 \\
 &\equiv 0 \pmod{11} \\ 
 \end{align*}
 より, $(42899634253)_{12}$ は $(11)_{10}$ で割り切れる.
 \item[3]  \mbox{} \\
 %%解いた
 \begin{align*}
 (3A6F2B1)_{16} &= (3 \times 16^{6} + 10 \times 16^{5} + 6 \times 16^{4} \\
  & + 15 \times 16^{3} + 2 \times 16^{2} + 11 \times 16^{1} + 1)_{10}\\
 \end{align*}
 また, $(16)_{10}^{k} \equiv 1 \pmod{15}$ より
 \begin{align*}
 (3A6F2B1)_{16}
 &\equiv (3 + 10 +6 + 15 + 2 + 11 + 1)_{10} \\
 &\equiv (48)_{10} \\
 &\equiv (3)_{10} \pmod{15} \\ 
 &\equiv (3)_{16} \pmod{F} \\
 \end{align*}
 \item[4]  \mbox{} \\
 \item[5]  \mbox{} \\
 \item[6]  \mbox{} \\
 \item[7]  \mbox{} \\
 \item[8]  \mbox{} \\
 %%追加説明あり
 \item[9]  \mbox{} \\
 \item[10]  \mbox{} \\
 \item[Option] $7$ の倍数の判定法の証明. \\
	\begin{align*}
		\begin{cases}
		N = 10a + b & \\
		M = a - 2b &
		\end{cases}
	\end{align*}
	とすると,
	\[ M \equiv 0 \pmod{7} \Leftrightarrow \exists k \in \mathbb{Z}~s.t.~a - 2b = 7k \]
	のとき,
	\begin{align*}
		N &= 10(7k + 2b) + b\\
		&= 7 \cdot 10k + 21b \\
		&= 7(10k + 2b) \\
		&\equiv 0 \pmod{7}
	\end{align*}
	より, $7|(a-2b)$, $7|(10a+b)$ となるので, 題意を満たす. 
\end{enumerate}
\pagebreak

\section*{問題3 - 2}
\begin{enumerate}
 \item[1]  \mbox{} \\
 %%これのみはだめ
 \item[2]  \mbox{} \\
 %%解いた
 $\left\{ 59, 162, -353, 107, 77, -50, 116 \right\}$ より,
 \begin{align*}
 59 &\equiv 3 \pmod{7} \\
 162 &\equiv 2 \pmod{7} \\
 -353 &\equiv 4 \pmod{7} \\
 107 &\equiv 2 \pmod{7} \\
 77 &\equiv 0 \pmod{7} \\
 -50 &\equiv 6 \pmod{7} \\
 116 &\equiv 4 \pmod{7} 
 \end{align*}
 より, $7$ を法とする\underline{完全代表系とはならない}.
 \item[3]  \mbox{} \\
 %%解いた
 $\left\{ -141, 65, 103, 70, -6, 199, 32 \right\}$ より,
 \begin{align*}
 -141 &\equiv 6 \pmod{7} \\
 65 &\equiv 2 \pmod{7} \\
 103 &\equiv 5 \pmod{7} \\
 70 &\equiv 0 \pmod{7} \\
 -6 &\equiv 1 \pmod{7} \\
 199 &\equiv 3 \pmod{7}
 \end{align*}
 より, $7$ を法とする\underline{完全代表系とはなる}.
 \item[4]  \mbox{} \\
 \item[5]  \mbox{} \\
\end{enumerate}
\pagebreak

\section*{問題3 - 3}
\begin{enumerate}
 \item[1]  \mbox{} \\
 %%わからんからmath-nakagawaおせーて
 \item[2]  \mbox{} \\
 \item[3]  \mbox{} \\
 \item[4]  \mbox{} \\
 %%reduce
 \item[5]  \mbox{} \\
 \item[6]  \mbox{} \\
 %%reduce
 \item[7]  \mbox{} \\
 %%わからんーーmath-nakagawaおせーて。あのね、さいごどうすればいいかわからん
 \begin{align*}
  \begin{cases}
  x \equiv c_1 \pmod{2} & \\
  x \equiv c_2 \pmod{3} & \\
  x \equiv c_3 \pmod{5} & \\
  x \equiv c_4 \pmod{7}
  \end{cases}
 \end{align*}
 より, 第 $1$ 式を満たす整数は,
 \[ x = c_1 + 2r~ (A) \]
 の形である. これが第 $2$ 式を満たすように $r$ を定める. \\
 $(A)$ を第 $2$ 式に代入すると,
 \[ c_1 + 2r \equiv c_2 \pmod{3} \] 
 であるから, そのためには $d_1 \equiv c_1 - c_2 \pmod{3}$ として $r$ を
 \[ 2r \equiv d_1~ (B) \]
 を満たすようにとればよい. $(2, 3) = 1$ より, $(B)$ の解は $3$ を法としてただ $1$ つ存在する. それを $r = r_1 \pmod{3}$ とすると, $(B)$ を満たす整数は,
 \[ r = r_1 + 3s \]
 の形である. したがってこれを $(A)$ に代入し, $x_2 = c_1 + 2r_1$ とおくと,
 \[ x = x_2 + 2 \times 3 \times s = x_2 + 6s \]
 が 第 $1$ 式, 第 $2$ 式をともに満たす整数となる. \\
 次にこれが第 $3$ 式を満たすように $s$ を定める. そして上と同様にして 第 $1$ 式から第 $3$ 式までを満たす整数が, 
 \[ x = x_3 + 2 \times 3 \times 5 \times t = x_3 + 30t \] 
 の形でえられる. ($x_3 = x_2 + 6s_1$)\\
 さらにこれが第 $4$ 式を満たすように $t$ を定める. そして上と同様にして 第 $1$ 式から第 $4$ 式までを満たす整数が, 
 \[ x = x_4 + 2 \times 3 \times 5 \times 7 \times u = x_4 + 210u \] 
 の形でえられる. ($x_4 = x_3 + 30t_1$) \\
 したがって, 求める解は,
 \begin{align*}
 x
 &\equiv x_4 \pmod{210}\\
 &\equiv x_3 + 30t_1 \pmod{210} \\
 &\equiv x_2 + 6s_1 + 30t_1 \pmod{210} \\
 &\equiv c_1 + 2r_1 + 6s_1 + 30t_1 \pmod{210} \\
 \end{align*}
 \item[8]  \mbox{} \\
 \item[9]  \mbox{} \\
 \item[10]  \mbox{} \\
 \item[11]  \mbox{} \\
\end{enumerate}
\pagebreak

\section*{問題3 - 4}
\begin{enumerate}
 \item[1]  \mbox{} \\
 %%解いた
 \begin{itemize}
 \item[(i)]
 \begin{align*}
 3^{47} &\equiv (3^3)^{15} \times 3^2 \\
      &\equiv 4^{15} \times 3^2 ~(\because 3^3 \equiv 27 \equiv 4 \pmod{23} \mbox{より}) \\
      &\equiv (4^3)^5 \times 3^2 \\
      &\equiv 18^5 \times 3^2 ~(\because 4^3 \equiv 64 \equiv 18 \pmod{23} \mbox{より}) \\
      &\equiv (-5)^5 \times 3^2 ~(\because 18 \equiv -5 \pmod{23} \mbox{より}) \\
      &\equiv \left\{(-5)^2 \right\}^2 \times (-5) \times 3^2 \\
      &\equiv 2^2 \times (-5) \times 3^2 ~(\because (-5)^2 \equiv 25 \equiv 2 \pmod{23} \mbox{より}) \\
      &\equiv -19 \\
      &\equiv 4 \pmod{23}
 \end{align*}
 \item[(ii)]
 \begin{align*}
 7^{1000} &\equiv (7^2)^{500} \\
      &\equiv 1^{500} ~(\because 7^2 \equiv 49 \equiv 1 \pmod{24}\mbox{より}) \\
      &\equiv 1 \pmod{24}
 \end{align*}
 \end{itemize}
 \item[2]  \mbox{} \\
 \item[3]  \mbox{いずれもMathematicaで素因数分解の計算を行った} \\
 %%ときましたよ。Mathematicaファイル参照
 \begin{itemize}
 \item[(i)]
 $n = 127281 = 3 \times 7 \times 11 \times 19 \times 29 $ より,
 \begin{align*}
 \varphi(n) &= 127281 \left(1-\frac{1}{3}\right) \left(1-\frac{1}{7}\right)
   \left(1-\frac{1}{11}\right) \left(1-\frac{1}{19}\right)
   \left(1-\frac{1}{29}\right) \\
 &= 60480
 \end{align*}
 また,
 \[\mu(n) = (-1)^5 = -1 \]
 \item[(ii)]
 $n = 18538 = 2 \times 13 \times 23 \times 31$ より,
 \begin{align*}
 \varphi(n) &= 18538 \left(1-\frac{1}{2}\right) \left(1-\frac{1}{13}\right)
   \left(1-\frac{1}{23}\right) \left(1-\frac{1}{31}\right) \\
 &= 7920
 \end{align*}
 また,
 \[\mu(n) = (-1)^4 = 1 \]
 \item[(iii)]
 $n = 200655 = 3^2 \times 5 \times 7^3 \times 13$ より,
 \begin{align*}
 \varphi(n) &= 200655 \left(1-\frac{1}{3}\right) \left(1-\frac{1}{5}\right)
   \left(1-\frac{1}{7}\right) \left(1-\frac{1}{13}\right) \\
 &= 84672
 \end{align*}
 また,
 \[\mu(n) = 0 \]
 \end{itemize}
 \item[4]  \mbox{} \\
 \item[5]  \mbox{} \\
 \item[6]  \mbox{} \\
 \item[7]  \mbox{} \\
 \item[8]  \mbox{} \\
 %%結果をReduceに用いる
 \item[9]  \mbox{} \\
 \item[10]  \mbox{} \\
 \item[RSA]  \mbox{} \\
\end{enumerate}
\pagebreak


%%%%%%%%%%%%%%%%%%%%%%%%%%%%%%%%%%%%%%%%%%%%%%%
%%元の話か。ここらへんからは任せたぞ、!!!!!!!!(人任せ)
%%%%%%%%%%%%%%%%%%%%%%%%%%%%%%%%%%%%%%%%%%%%%%%
\section*{問題3 - 5}
\begin{enumerate}
 %\item[1]  \mbox{} \\
 %%除外
 \item[2]  \mbox{} \\
 \item[3]  \mbox{} \\
 \item[4]  \mbox{} \\
 \item[5]  \mbox{} \\
 \item[6]  \mbox{} \\
\end{enumerate}
\pagebreak

\section*{問題3 - 6}
\begin{enumerate}
 \item[1]  \mbox{} \\
 \item[2]  \mbox{} \\
 \item[3]  \mbox{} \\
 \item[4]  \mbox{} \\
 \item[5]  \mbox{} \\
 \item[6]  \mbox{} \\
 \item[7]  \mbox{} \\
 \item[8]  \mbox{} \\
 \item[9]  \mbox{} \\
 \item[10]  \mbox{} \\
 \item[11]  \mbox{} \\
\end{enumerate}
\pagebreak

\section*{問題3 - 6}
\begin{enumerate}
 \item[FGLM]  \mbox{} \\
 \item[定理証明]  \mbox{} \\
\end{enumerate}

\end{document}
